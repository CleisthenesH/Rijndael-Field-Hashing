% Copyright 2023 Kieran W Harvie. All rights reserved.

\chapter{Advanced Encryption Standard}
As alluded to in the introduction one use of the Rijndael field is the Advanced Encryption Standard (AES).
AES has two types of components,
the transforms and the key schedule.
The transforms act on a 4 by 4 matrix of $GF(2^8)$ elements called the state:
\[
\begin{bmatrix}
	m_{0,0}&m_{0,1}&m_{0,2}&m_{0,3}\\
	m_{1,0}&m_{1,1}&m_{1,2}&m_{1,3}\\
	m_{2,0}&m_{2,1}&m_{2,2}&m_{2,3}\\
	m_{3,0}&m_{3,1}&m_{3,2}&m_{3,3}\\
\end{bmatrix}
\]
The key schedule takes an original cipher key and outputs round keys used by a transform:
\[
\begin{bmatrix}
	k_{0,0}&k_{0,1}&k_{0,2}&k_{0,3}\\
	k_{1,0}&k_{1,1}&k_{1,2}&k_{1,3}\\
	k_{2,0}&k_{2,1}&k_{2,2}&k_{2,3}\\
	k_{3,0}&k_{3,1}&k_{3,2}&k_{3,3}\\
\end{bmatrix}
\]

The basis of AES is the Rijndael field and some matrix operations,
meaning we inadvertently have all tools necessary to give an overview.
Not sure I will implment this in C.

\section{Rijndael S-box}
The Rijndael S-box is the main source of non-linearlity in the AES and is used in both the key schedule and a transform.
It's just a function that takes and returns an element of $GF(2^8)$,
which is why the S stands for substitution.
\\

First,
for non-zero input $m$,
take the binary expansion of its reciprocal:
\[m^{-1}=b_7b_6b_5b_4b_3b_2b_1b_0\]
(When $m=0$ set all $b_n=0$).
\\

Then construct a new byte with the following affine transform,
arithmetic is in $GF(2)$:
\[
\begin{bmatrix}
b_0'\\ b_1'\\ b_2'\\ b_3'\\ b_4'\\ b_5'\\ b_6'\\ b_7'\\
\end{bmatrix}
=
\begin{bmatrix}
	1&0&0&0&1&1&1&1\\
	1&1&0&0&0&1&1&1\\
	1&1&1&0&0&0&1&1\\
	1&1&1&1&0&0&0&1\\
	1&1&1&1&1&0&0&0\\
	0&1&1&1&1&1&0&0\\
	0&0&1&1&1&1&1&0\\
	0&0&0&1&1&1&1&1\\
\end{bmatrix}
\begin{bmatrix}
b_0\\ b_1\\ b_2\\ b_3\\ b_4\\ b_5\\ b_6\\ b_7\\
\end{bmatrix}
+
\begin{bmatrix}
1\\1\\0\\0\\0\\1\\1\\0\\
\end{bmatrix}
\]

Finally,
output the element with the binary expansion of:
\[b_7'b_6'b_5'b_4'b_3'b_2'b_1'b_0'\]

The Rijndael S-box is a one-to-one map so can easily and unambiguously be inverted by look-up.
But for those interested,
the inverse affine transform is given bellow:
\[
\begin{bmatrix}
b_0\\ b_1\\ b_2\\ b_3\\ b_4\\ b_5\\ b_6\\ b_7\\
\end{bmatrix}
=
\begin{bmatrix}
	0&0&1&0&0&1&0&1\\
	1&0&0&1&0&0&1&0\\
	0&1&0&0&1&0&0&1\\
	1&0&1&0&0&1&0&0\\
	0&1&0&1&0&0&1&0\\
	0&0&1&0&1&0&0&1\\
	1&0&0&1&0&1&0&0\\
	0&1&0&0&1&0&1&0\\
\end{bmatrix}
\begin{bmatrix}
b_0'\\ b_1'\\ b_2'\\ b_3'\\ b_4'\\ b_5'\\ b_6'\\ b_7'\\
\end{bmatrix}
+
\begin{bmatrix}
1\\0\\1\\0\\0\\0\\0\\0\\
\end{bmatrix}
\]
And you can invert the Rijndael S-box by applying this transform then reciprocating any non zero element.

\section{Transforms}
\subsubsection{SubBytes}
The output of this SubBytes is a block where each element is substituted with the output of this process.

\subsubsection{ShiftRows}
\[
\begin{bmatrix}
	m_{0,0}&m_{0,1}&m_{0,2}&m_{0,3}\\
	m_{1,0}&m_{1,1}&m_{1,2}&m_{1,3}\\
	m_{2,0}&m_{2,1}&m_{2,2}&m_{2,3}\\
	m_{3,0}&m_{3,1}&m_{3,2}&m_{3,3}\\
\end{bmatrix}
\rightarrow
\begin{bmatrix}
	m_{0,0}&m_{0,1}&m_{0,2}&m_{0,3}\\
	m_{1,1}&m_{1,2}&m_{1,3}&m_{1,0}\\
	m_{2,2}&m_{2,3}&m_{2,0}&m_{2,1}\\
	m_{3,3}&m_{3,0}&m_{3,1}&m_{3,2}\\
\end{bmatrix}
\]
\subsubsection{MixColumns}
\[
\begin{bmatrix}
	m_{0,0}&m_{0,1}&m_{0,2}&m_{0,3}\\
	m_{1,0}&m_{1,1}&m_{1,2}&m_{1,3}\\
	m_{2,0}&m_{2,1}&m_{2,2}&m_{2,3}\\
	m_{3,0}&m_{3,1}&m_{3,2}&m_{3,3}\\
\end{bmatrix}
\rightarrow
\begin{bmatrix}
	2&3&1&1\\
	1&2&3&1\\
	1&1&2&3\\	
	3&1&1&2\\
\end{bmatrix}
\begin{bmatrix}
	m_{0,0}&m_{0,1}&m_{0,2}&m_{0,3}\\
	m_{1,0}&m_{1,1}&m_{1,2}&m_{1,3}\\
	m_{2,0}&m_{2,1}&m_{2,2}&m_{2,3}\\
	m_{3,0}&m_{3,1}&m_{3,2}&m_{3,3}\\
\end{bmatrix}
\]
\subsubsection{AddRoundKey}
\[
\begin{bmatrix}
	m_{0,0}&m_{0,1}&m_{0,2}&m_{0,3}\\
	m_{1,0}&m_{1,1}&m_{1,2}&m_{1,3}\\
	m_{2,0}&m_{2,1}&m_{2,2}&m_{2,3}\\
	m_{3,0}&m_{3,1}&m_{3,2}&m_{3,3}\\
\end{bmatrix}
\rightarrow
\begin{bmatrix}
	m_{0,0}&m_{0,1}&m_{0,2}&m_{0,3}\\
	m_{1,0}&m_{1,1}&m_{1,2}&m_{1,3}\\
	m_{2,0}&m_{2,1}&m_{2,2}&m_{2,3}\\
	m_{3,0}&m_{3,1}&m_{3,2}&m_{3,3}\\
\end{bmatrix}
+
\begin{bmatrix}
	k_{0,0}&k_{0,1}&k_{0,2}&k_{0,3}\\
	k_{1,0}&k_{1,1}&k_{1,2}&k_{1,3}\\
	k_{2,0}&k_{2,1}&k_{2,2}&k_{2,3}\\
	k_{3,0}&k_{3,1}&k_{3,2}&k_{3,3}\\
\end{bmatrix}
\]
\section{Key Schedule}
