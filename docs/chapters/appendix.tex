% Copyright 2023 Kieran W Harvie. All rights reserved.

\chapter{Appendix}
\section{Generator Table}
\section{Side-Channel}
\label{appx:side-channel}
I've said in a few places that using generators to simplify field arithmetic makes the implementation susceptible to side-channel attacks.
But how?
And what is a side-channel attack?
\\

A side-channel is something about the implementation of an algorithm or protocol that lets an attacker get information,
rather than something inherently wrong with the algorithm or protocol itself. 
\\

The easiest to explain side-channel is the timing.
Consider a guarded door and to get access you have to tell the guard a password.
You tell the guard two different incorrect passwords but the guard rejects one immediately but waits till you're half way through for the second one.
You might reasonably conclude that first password is wholly wrong but that the first half of the second one is correct.
By doing this enough times and keeping track of the time the guard rejects the password you can probably figure out the password.
You guessing the password has nothing to do with the cryptography of a password but the implementation by the guard,
instead they should have waited till you've said the whole password before rejecting it, 
and hence is a side-channel attack.
\\

Using a look-up table to compute field operations opens you up to a caching side-channel attack.
\\

Because the look-up table is to large to be stored in the CPU registers it will be accessed from memory.
Accessing the table from memory means it's likely to be cashed.
The table being cached means someone with the ability to monitor cache access can see what field elements you're using,
and maybe work backwards to get some of the string key that was hashed.
Hence leaking the string.
\\

This makes using look-up tables unsuitable for application where the string keys are sensitive information.
\\

Before closing this topic I want to bring up that side-channels can include the physical properties of computers as well.
Consider a naïve attempt at stopping a timing attack where the guard waits to reject the password but is still noticeably paying less attention.
An attacker can still guess the timing from when the guard stops paying attention.
Similarly idling a computer till a set amount of time passes for all password validation attempts might cause a change in power consumption or a noticeable reduction in cooling. 
Hence even power-draw or acoustic properties of a computer are possible side-channels for attacks.

\section{Matrix Arithmetic}
\subsection{Diagonalization}
The description of Gaussian Elimination and the claim that it works with any invertible matrix may make some readers that half remember a linear algebra course nervous.
This is because process described is similar to another process, diagonalization, which is not guaranteed to work just because the input matrix is invertible.
\\

A diagonal matrix is one where all non-diagonal elements are zero and transcribed as $\Diag$:
\[\begin{bmatrix}1&0&0&0\\0&3&0&0\\0&0&2&0\\0&0&0&7\end{bmatrix} = \Diag(1,3,2,7)\]
Such matrices are important they are the same as the matrix for the elementary row operation of scaling the rows,
making matrix multiplication invoking them conceptually and computationally easy.
For example:
\[\Diag(a_0,a_1,\dots a_n)\Diag(b_0,b_1,\dots b_n) = \Diag(a_0b_0,a_1b_1,\dots a_nb_n)\]
\\

Diagonalization is the process of taking a matrix $M$ and finding a diagonal matrix $D$ and invertible matrix $P$ such that:
\footnote{Often equivalently written as $M = PDP^{-1}$}
\[MP = PD\]
And this process is important because it enables the simplification of calculations by canceling $P$'s and utilizing diagonal matrix multiplication:
\begin{equation*}
\begin{aligned}
 M^3 =& (PDP^{-1})^3 \\
 =& (PDP^{-1})(PDP^{-1})(PDP^{-1}) \\
 =& PD(P^{-1} P)D(P^{-1} P)DP^{-1} \\
 =& PD^3P^{-1} \\
 =& P\Diag(a_k)^3P^{-1} \\
 =& P\Diag(a_k^3)P^{-1} \\
\end{aligned}
\end{equation*}
\\

But unfortunately diagonalization isn't guaranteed for invertible matrices.
The conditions of when it works involves something called eigenvalues,
which are outside the scope of this document.
But what is concerning is how similar this looks to Gaussian Elimination,
which given $MA =B$ involves finding an invertible $P$ such that:
\[M'A =PMA = PB\]
Where $M'$ is $M$ after some elementary row operations.
What is important to notice is that in diagonalization $P$ is on the right side of $M$ and the left side of $D$.
But during Gaussian Elimination $P$ is on the left side of both $M$ and $B$.
Since matrix multiplication isn't commutative this makes the processes very different and lays nerves to rest. 
\\

Readers interesting in learning more about diagonalization and it's uses are encouraged to first loop up eigenvalues and then "The Jordan Normal Form".

\section{Polynomial Arithmetic}
