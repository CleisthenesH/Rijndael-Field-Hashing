% Copyright 2023 Kieran W Harvie. All rights reserved.

\chapter{Introduction}
General introduction
This is an accompanying document to the repository.

It is meant to approachable to someone that has freshman understand of math or programing,
and hopefully work as an instruction to the other.

And maybe even an ambassador for abstract algebra and cryptography for people that are interested in those things but need practical applications to make them stick.

\section{Structure of the Repository}

\section{Hashing 101}
For our purposes a hash function is an any function that inputs a string and outputs a number.
The most direct example is converting the first character to it alphabetical number (A=$1$, B=$2$, etc.):
\[\hash(\text{Apple}) = 1,\quad \hash(\text{Orange}) = 15\]

There are two main applications for hashing:
The first is performance,
as operations on strings may be more expensive then on numbers so an efficient and well designed hash may allow us to work better on the numbers instead. 
The second is cryptography,
that is the practice of communicating secrets,
where we want to hash the string and communicate the number in hopes of keeping the string secrete.
\\

There is a clear tension between these two applications.
For performance you'd want as much information about the initial string easily assessable in the hash.
But for cryptography you'd want as much information hidden as possible while still allowing communication. 
\\

This document aims to serve as an introduction,
so wont optimize for either application,
but I bring them up this early to wet the readers appetite about why they might care about hashing.
{\textbf{ But also to state that the reader should not apply any algorithm here for cryptographic purposes without consulting a professional first.}}
At least one technique used makes the algorithm susceptible to a \hyperref[appx:side-channel]{side-channel attack},
and even if I avoided that technique there would be another one I'm ignorant of.

\subsection{Hashing Terminology}
With applications and warning asides there are some more things the reader should know.
First is that in common use the term hash can mean both the hash function and the resulting number from that function.

Secondly there are four general terms worth knowing:

{\textbf{Collision:}} 
A collision is when two strings have the same hash.
This is broadly undesirable since any operation we want to preform on the hash has to work for both strings.

{\textbf{Hash Table:}}
A common application of hash functions in environments where accessing a look-up table with string keys are inefficient compared to number keys.
We basically make a look-up table with number keys and when asked to look-up a string key we hash it first and use the result as the integer key.
This will require us to deal with collisions as they come up.

{\textbf{Perfect Hash Function:}}
A perfect hash for a given set of strings is hash function where none of strings collide.

{\textbf{Minimal Perfect Hash Function:}}
A minimal perfect hash for a given set is a perfect hash function where all outputs of the string are less then or equal to the size of the set.
In terms of features,
but not necessarily performance,
this is the ideal hash function as there are no collisions to deal with and no wasted numbers taking up space.

\section{Field Theory 101}
Rational numbers
finite fields
define the term "field operations"
MAKE SURE TO USE THE APPENDIX!!

\subsection{Finite Fields}
history of Galois (wasn't understood in his time not helped by his duel)
prime powers
unique
polynomials

To understand why prime numbers might be imporant consider the following:
Assume there is $x$ such that $px=1 \mod pq$ then we have $x < q$ else we set $x -= q$ till its true.
But $x < q$ implies $px < pq$ meaning we never have to take a remainder hence $px = 1$ but $p>1$ and $x>1$ means $px> 1$ a contraction.

This type of argument can be use to prove a Bézout's identity

\section{The Rijndael Field}
notation and generators 

you can ignore all this theory and just have two operation on bytes that 
